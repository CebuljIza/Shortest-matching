\documentclass[a4paper, 11pt]{article}
% ukazi za delo s slovenscino -- izberi kodiranje, ki ti ustreza
\usepackage[slovene]{babel}
\usepackage[utf8]{inputenc}
\usepackage[T1]{fontenc}
\usepackage[utf8]{inputenc}
\usepackage{amsmath,amssymb,amsfonts}
\usepackage{url}
\usepackage[normalem]{ulem}
\usepackage[dvipsnames,usenames]{color}
\usepackage{graphicx}
\usepackage{float}
\usepackage{verbatim}

% okolje za slike
%\graphicspath{}


% ukazi za matematicna okolja
%\theoremstyle{definition} % tekst napisan pokoncno
%\newtheorem{definicija}{Definicija}[section]
%\newtheorem{primer}[definicija]{Primer}
%\theoremstyle{remark}
%\newtheorem*{remark}{Opomba}


%\renewcommand\endprimer{\hfill$\diamondsuit$}


%\theoremstyle{plain} % tekst napisan posevno
%\newtheorem{lema}[definicija]{Lema}
%\newtheorem{izrek}[definicija]{Izrek}
%\newtheorem{trditev}[definicija]{Trditev}
%\newtheorem{posledica}[definicija]{Posledica}


% za stevilske mnozice uporabi naslednje simbole
\newcommand{\R}{\mathbb R}
\newcommand{\N}{\mathbb N}
\newcommand{\Z}{\mathbb Z}
\newcommand{\C}{\mathbb C}
\newcommand{\Q}{\mathbb Q}

% ukaz za slovarsko geslo
\newlength{\odstavek}
\setlength{\odstavek}{\parindent}
\newcommand{\geslo}[2]{\noindent\textbf{#1}\hspace*{3mm}\hangindent=\parindent\hangafter=1 #2}

% naslednje ukaze ustrezno popravi
\newcommand{\program}{Finančna matematika 1.~stopnja} % ime studijskega programa: Matematika/Finan"cna matematika
\newcommand{\imeavtorja}{Iza Čebulj, Barbara Pal} % ime avtorja
\newcommand{\naslovdela}{Najcenejše prirejanje v ravnini}
\newcommand{\letnica}{2022} 
\newcommand{\predmet}{Finančni praktikum}

% vstavi svoje definicije ...

%%%%%%%%%%%%%%%%%%%%%%%%%%%%%%%%%%%%%%%%%%%%%%%


\begin{document}

\thispagestyle{empty}
\begin{center}
\begin{minipage}{0.75\linewidth}
    \centering
    {\Large Univerza v Ljubljani \\ \program}
    \\
    \vspace{3cm}

    {\uppercase{\Large \textbf{\naslovdela}}} \\ \predmet\\
    \vspace{3cm}

    {\Large \imeavtorja\par}
    \vspace{9cm}

\end{minipage}
\end{center}

\noindent{\large
Ljubljana, \letnica}
\pagebreak

\thispagestyle{empty}
\tableofcontents
\listoffigures
\pagebreak

\section{Navodilo}
Naj bo $P$ množica z $2n$ točkami na ravnini. Najcenejše prirejanje za $P$ je množica daljic z minimalno skupno ceno, za katere velja da je vsaka točka iz množice $P$ končna točka natanko ene daljice, cena pa je enaka razdalji med končnima točkama daljice. \\
Formuliraj problem kot CLP. Preko eksperimentov poišči pričakovano vrednost celotne cene najcenejšega prirejanja, ko so točke izbrane naključno v enotskem kvadratu, enotskem krogu in enakostraničnem trikotniku.
Ugotovi, ali se vrednost z večanjem $n$ povečuje ali zmanjšuje. \\
Obravnavaj tudi problem, ko je množica $P$ sestavljena iz $n$ rdečih in $n$ modrih točk in so končne točke daljic različnih barv - dvobarvno najcenejše prirejanje. 
Primerjaj vrednost v primeru, ko točke ločimo po barvah in v primeru, ko jih ne.  

\section{Reševanje osnovnega problema najcenejšega prirejanja}
\subsection{Opis problema}
Množici $P$ z $2n$ točkami lahko priredimo neusmerjen graf $G(P,E),$ oziroma samo $G.$
Množica vozlišč grafa $G$ je kar množica $P,$ množica povezav v grafu $E$ pa so neurejeni pari vozlišč $(u,v),$ za katere velja $u,~v \in P$ in $u \neq v.$ 
Cena povezave je razdalja $d(u,v)$ med vozliščema $u$ in $v.$ \\
Popolno prirejanje na grafu $G$ oziroma v množici $P$ je taka množica povezav $M,$ za katero velja, da vsako vozlišče v $P$ sovpada z natanko eno povezavo v $M$.
Velikost popolnega prirejanja v množici velikosti $2n$ je $n.$ 
Ceno prirejanja definiramo kot $\sum_{(u,v) \in M} d(u,v),$ kar je vsota cen vseh povezav v $M.$ \\
Radi bi poiskali \emph{najcenejše popolno prirejanje} in njegovo ceno za različne $n.$

\subsection{Zapis problema kot celoštevilski linearni program}

\subsection{Programiranje rešitev in eksperimentiranje}

\subsection{Analiza rezultatov}
\subsubsection*{Izbira točk v enotskem kvadratu}

\subsubsection*{Izbira točk v enotskem krogu}

\subsubsection*{Izbira točk v enakostraničnem trikotniku}

\subsubsection*{Časovna odvisnost algoritma}


\section{Dvobarvno najcenejše prirejanje}
\subsection{Opis problema}
Množico $P$ sestavljata množica $n$ rdečih točk, $R,$ in množica $n$ modrih točk $B,$ $P=R \cup B.$
V tem primeru je $G(P,E)$ dvodelen graf z lastnostjo, da med dvema točkama obstaja povezava, če in samo če sta različnih barv.
Cene povezav $(u,v)$ so tako kot v osnovnem primeru razdalje med vozlišči, $d(u,v).$
Spet iščemo najcenejše popolno prirejanje in njegovo ceno.

\subsection{Programiranje rešitev in eksperimentiranje}

\subsection{Analiza rezultatov}


\section{Zaključek}
% primerjava rezultatov obeh algoritmov

% \section{Literatura/Viri} ?

\end{document}